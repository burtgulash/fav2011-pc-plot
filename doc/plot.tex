\documentclass[11pt]{article}
\usepackage[czech]{babel}
\usepackage[utf8]{inputenc}
\usepackage{a4wide}

\usepackage[pdf]{pstricks}

\title{Vizualizace grafu matematické funkce}
\author{Tomáš Maršálek}
\date{\today}

\begin{document}
\maketitle

\section{Zadání}
Naprogramujte v ANSI C přenositelnou1 konzolovou aplikaci, která jako vstup
načte z parametru na příkazové řádce matematickou funkci ve tvaru $y = f(x)$,
provede její analýzu a vytvoří soubor ve formátu PostScript s grafem této
funkce na zvoleném definičním oboru.

\section{Analýza úlohy}
Abychom mohli zobrazit graf funkce jako například x\^2 + sin(x), musíme být
nejdříve schopni vyhodnotit tuto funkci numericky. Tedy musíme ve výrazu
rozpoznat jednotlivá čísla, proměnné nebo operátory a rozpoznat jejich význam.
Standardním postupem při parsování vstupního řetězce je rozdělení úlohy na
několik po sobě následujících částí tak, aby se případné chyby ve vstupu
odhalily v příslušné části a nepropagovaly se do následující části. Každá část
se tedy stará pouze o chyby, které se jí týkají.

\subsection{Lexikální analýza}
Místo abychom při vyhodnocování výrazu postupně hledali čísla nebo operátory,
ponecháme tuto úlohu specializovanému nástroji, takzvanému scanneru.
Nadefinujeme jednotlivé symboly a necháme si od scanneru vrátit rozpoznané
symboly nebo případně chybu, pokud nebyl symbol rozpoznán. Význam celé této
fáze je v zjednodušení nadcházející fáze, která je složitější, a už se nebude
muset zabývat problémy typu jestli je symbol skutečně číslo nebo jestli je
symbol $\sin$ ve skutečnosti $\sinh$ a podobně.

\subsection{Syntaktická a sémantická analýza}
Tyto dvě fáze jsou zde kombinované v jednu. Zde vytvoříme přeloženou datovou
strukturu, se kterou bude vyhodnocování výrazu víceméně triviální. Syntaktická
analýza najde syntaktické chyby, jako například dvě čísla nebo znaky
následující po sobě, chybějící závorky, a podobně. Jednotlivým symbolům přiřadíme jejich význam, tj. číslům dát jejich numerickou hodnotu a operátorům 
jejich příslušnou unární nebo binární funkci. 

\subsection{Postfixová notace}
Výsledná datová struktura je posloupnost symbolů v postfixové nebo také
reverzní polské notaci. Výhoda tohoto zápisu oproti klasické infixové notaci
je absence závorek a jednoduchost vyhodnocení výrazu, při vyhodnocování totiž
vždy když narazíme na operátor, máme jistotu, že všechny předcházející znaky, 
které operátor vyžaduje, jsou čísla.

Příklad infixové a postfixové notace.
\begin{center}
\begin{tabular}{|c|c|}
\hline
infix & postfix \\
\hline \hline
$1~+~1$ & $1~1~+$ \\
$1~+~2~*~3$ & $1~2~3~*~+$ \\
$(1~+~2)~*~3$ & $1~2~+~3~*$ \\
$x~*~\sin(x^2)$ & $x~x~2~\verb|^|~\sin~*$ \\
\hline
\end{tabular}
\end{center}

Metoda na převedení infixového do postfixového zápisu je např. Dijkstrův
Shunting yard algoritmus.


\subsection{Shunting yard}
Při konverzi musíme mít na paměti přednosti jednotlivých operátorů, tedy výraz
$1 + 2 * 3$ se musí přeložit jako $1~2~3~*~+$ a ne jako $1~2~+~3~*$.  Dále také
levou nebo pravou asociaci nekomutativních operátorů, pokud ve výrazů chybí
závorky.  Například operátor odčítání upřednostňuje levé uzávorkování ($1 - 2 -
3 = ((1 - 2) - 3)$) oproti umocňování, které naopak upřednostňuje uzávorkování
zprava ($2~\verb|^|~3~\verb|^|~4 = (2~\verb|^|~(3~\verb|^|~4)) = 2^{3^4}$).

Algoritmus používá pomocný zásobník na odkládání operátorů, dokud nemá jistotu
o jejich správném umístění v postfixu. Čteme infixový výraz zleva doprava po
jednotlivých symbolech a vždy když narazíme na symbol typu číslo nebo proměnná,
pouze ho uložíme do výsledné výstupní fronty. Pokud narazíme na operátor,
všechny operátory, které dosud leží nahoře v zásobníku a vážou se těsněji (mají
vyšší precedenci) než právě vytažený symbol, můžeme přidat do výsledné fronty.
Právě vytažený symbol pak pouze vložíme na zásobník. Až nám dojdou všechny
symboly, pouze vytáhneme všechny operátory ze zásobníku a v tomto pořadí je
přidáme do výsledku. \\

Jako příklad mějme výraz 1 / 2~\verb|^|~3 + 4.
\begin{center}
\begin{tabular}{|c|l|l|r|}
\hline
symbol & výsledek & zásobník & akce \\
\hline
1 & 1 & & číslo, pouze přidáme do fronty \\
/ & 1 & / & dosud žádný operátor v zásobníku\\
2 & 1 2 & / & číslo, pouze přidáme do fronty \\
\verb|^| & 1 2 & / \verb|^| & / má slabší vazbu než \verb|^|, necháme být \\
3 & 1 2 3 & / \verb|^| & číslo, pouze přidáme do fronty \\
+ & 1 2 3 \verb|^| / & + & \verb|^| a / mají silnější vazbu než +, 
přidáme je do fronty\\
4 & 1 2 3 \verb|^| / 4 & + & číslo, pouze přidáme do fronty \\
  & 1 2 3 \verb|^| / 4 + & & konec, všechny operátory přidáme do fronty \\

\hline
\end{tabular}
\end{center}



\section{Implementace}
\section{Závěr}
\end{document}
